\documentclass{report}
\title{Homework 6}
\author{\textbf{Chirag Sachdev}}
\date{Week 6}
\usepackage{634format}
\usepackage{enumerate}
\usepackage{listings}
\usepackage{textcomp}
\usepackage{amsmath}
\usepackage{amssymb}
\usepackage{hyperref}
\usepackage{holtex}
\usepackage{holtexbasic}
% =====================================================================
%
% Macros for typesetting the HOL system manual
%
% =====================================================================

% ---------------------------------------------------------------------
% Abbreviations for words and phrases
% ---------------------------------------------------------------------

\newcommand\TUTORIAL{{\footnotesize\sl TUTORIAL}}
\newcommand\DESCRIPTION{{\footnotesize\sl DESCRIPTION}}
\newcommand\REFERENCE{{\footnotesize\sl REFERENCE}}
\newcommand\LOGIC{{\footnotesize\sl LOGIC}}
\newcommand\LIBRARIES{{\footnotesize\sl LIBRARIES}}
\usepackage{textcomp}

\newcommand{\bs}{\texttt{\char'134}} % backslash
\newcommand{\lb}{\texttt{\char'173}} % left brace
\newcommand{\rb}{\texttt{\char'175}} % right brace
\newcommand{\td}{\texttt{\char'176}} % tilde
\newcommand{\lt}{\texttt{\char'74}} % less than
\newcommand{\gt}{\texttt{\char'76}} % greater than
\newcommand{\dol}{\texttt{\char'44}} % dollar
\newcommand{\pipe}{\texttt{\char'174}}
\newcommand{\apost}{\texttt{\textquotesingle}}
% double back quotes ``
\newcommand{\dq}{\texttt{\char'140\char'140}}
%These macros were included by slind:

\newcommand{\holquote}[1]{\dq#1\dq}

\def\HOL{\textsc{Hol}}
\def\holn{\HOL}  % i.e. hol n(inety-eight), no digits in
                 % macro names is a bit of a pain; deciding to do away
                 % with hol98 nomenclature means that we just want to
                 % write HOL for hol98.
\def\holnversion{Kananaskis-11}
\def\holnsversion{Kananaskis~11} % version with space rather than hyphen
\def\LCF{{\small LCF}}
\def\LCFLSM{{\small LCF{\kern-.2em}{\normalsize\_}{\kern0.1em}LSM}}
\def\PPL{{\small PP}{\kern-.095em}$\lambda$}
\def\PPLAMBDA{{\small PPLAMBDA}}
\def\ML{{\small ML}}
\def\holmake{\texttt{Holmake}}

\newcommand\ie{\mbox{\textit{i{.}e{.}}}}
\newcommand\eg{\mbox{\textit{e{.}g{.}}}}
\newcommand\viz{\mbox{viz{.}}}
\newcommand\adhoc{\mbox{\it ad hoc}}
\newcommand\etal{{\it et al.\/}}
% NOTE: \etc produces wrong spacing if used between sentences, that is
% like here \etc End such sentences with non-macro etc.
\newcommand\etc{\mbox{\textit{etc{.}}}}

% ---------------------------------------------------------------------
% Simple abbreviations and macros for mathematical typesetting
% ---------------------------------------------------------------------

\newcommand\fun{{\to}}
\newcommand\prd{{\times}}

\newcommand\conj{\ \wedge\ }
\newcommand\disj{\ \vee\ }
\newcommand\imp{ \Rightarrow }
\newcommand\eqv{\ \equiv\ }
\newcommand\cond{\rightarrow}
\newcommand\vbar{\mid}
\newcommand\turn{\ \vdash\ } % FIXME: "\ " results in extra space
\newcommand\hilbert{\varepsilon}
\newcommand\eqdef{\ \equiv\ }

\newcommand\natnums{\mbox{${\sf N}\!\!\!\!{\sf N}$}}
\newcommand\bools{\mbox{${\sf T}\!\!\!\!{\sf T}$}}

\newcommand\p{$\prime$}
\newcommand\f{$\forall$\ }
\newcommand\e{$\exists$\ }

\newcommand\orr{$\vee$\ }
\newcommand\negg{$\neg$\ }

\newcommand\arrr{$\rightarrow$}
\newcommand\hex{$\sharp $}

\newcommand{\uquant}[1]{\forall #1.\ }
\newcommand{\equant}[1]{\exists #1.\ }
\newcommand{\hquant}[1]{\hilbert #1.\ }
\newcommand{\iquant}[1]{\exists ! #1.\ }
\newcommand{\lquant}[1]{\lambda #1.\ }

\newcommand{\leave}[1]{\\[#1]\noindent}
\newcommand\entails{\mbox{\rule{.3mm}{4mm}\rule[2mm]{.2in}{.3mm}}}

% ---------------------------------------------------------------------
% Font-changing commands
% ---------------------------------------------------------------------

\newcommand{\theory}[1]{\hbox{{\small\tt #1}}}
\newcommand{\theoryimp}[1]{\texttt{#1}}

\newcommand{\con}[1]{{\sf #1}}
\newcommand{\rul}[1]{{\tt #1}}
\newcommand{\ty}[1]{\textsl{#1}}

\newcommand{\ml}[1]{\mbox{{\def\_{\char'137}\texttt{#1}}}}
\newcommand{\holtxt}[1]{\ml{#1}}
\newcommand\ms{\tt}
\newcommand{\s}[1]{{\small #1}}

\newcommand{\pin}[1]{{\bf #1}}
% FIXME: for multichar symbols \mathit should be used.
\def\m#1{\mbox{\normalsize$#1$}}

% ---------------------------------------------------------------------
% Abbreviations for particular mathematical constants etc.
% ---------------------------------------------------------------------

\newcommand\T{\con{T}}
\newcommand\F{\con{F}}
\newcommand\OneOne{\con{One\_One}}
\newcommand\OntoSubset{\con{Onto\_Subset}}
\newcommand\Onto{\con{Onto}}
\newcommand\TyDef{\con{Type\_Definition}}
\newcommand\Inv{\con{Inv}}
\newcommand\com{\con{o}}
\newcommand\Id{\con{I}}
\newcommand\MkPair{\con{Mk\_Pair}}
\newcommand\IsPair{\con{Is\_Pair}}
\newcommand\Fst{\con{Fst}}
\newcommand\Snd{\con{Snd}}
\newcommand\Suc{\con{Suc}}
\newcommand\Nil{\con{Nil}}
\newcommand\Cons{\con{Cons}}
\newcommand\Hd{\con{Hd}}
\newcommand\Tl{\con{Tl}}
\newcommand\Null{\con{Null}}
\newcommand\ListPrimRec{\con{List\_Prim\_Rec}}


\newcommand\SimpRec{\con{Simp\_Rec}}
\newcommand\SimpRecRel{\con{Simp\_Rec\_Rel}}
\newcommand\SimpRecFun{\con{Simp\_Rec\_Fun}}
\newcommand\PrimRec{\con{Prim\_Rec}}
\newcommand\PrimRecRel{\con{Prim\_Rec\_Rel}}
\newcommand\PrimRecFun{\con{Prim\_Rec\_Fun}}

\newcommand\bool{\ty{bool}}
\newcommand\num{\ty{num}}
\newcommand\ind{\ty{ind}}
\newcommand\lst{\ty{list}}

% ---------------------------------------------------------------------
% \minipagewidth = \textwidth minus 1.02 em
% ---------------------------------------------------------------------

\newlength{\minipagewidth}
\setlength{\minipagewidth}{\textwidth}
\addtolength{\minipagewidth}{-1.02em}

% ---------------------------------------------------------------------
% Environment for the items on the title page of a case study
% ---------------------------------------------------------------------

\newenvironment{inset}[1]{\noindent{\large\bf #1}\begin{list}%
{}{\setlength{\leftmargin}{\parindent}%
\setlength{\topsep}{-.1in}}\item }{\end{list}\vskip .4in}

% ---------------------------------------------------------------------
% Macros for little HOL sessions displayed in boxes.
%
% Usage: (1) \setcounter{sessioncount}{1} resets the session counter
%
%        (2) \begin{session}\begin{verbatim}
%             .
%              < lines from hol session >
%             .
%            \end{verbatim}\end{session}
%
%            typesets the session in a numbered box.
% ---------------------------------------------------------------------

\newlength{\hsbw}
\setlength{\hsbw}{\textwidth}
\addtolength{\hsbw}{-\arrayrulewidth}
\addtolength{\hsbw}{-\tabcolsep}
\newcommand\HOLSpacing{13pt}

\newcounter{sessioncount}
\setcounter{sessioncount}{0}

\newenvironment{session}{\begin{flushleft}
 \refstepcounter{sessioncount}
 \begin{tabular}{@{}|c@{}|@{}}\hline
 \begin{minipage}[b]{\hsbw}
 \vspace*{-.5pt}
 \begin{flushright}
 \rule{0.01in}{.15in}\rule{0.3in}{0.01in}\hspace{-0.35in}
 \raisebox{0.04in}{\makebox[0.3in][c]{\footnotesize\sl \thesessioncount}}
 \end{flushright}
 \vspace*{-.55in}
 \begingroup\small\baselineskip\HOLSpacing}{\endgroup\end{minipage}\\ \hline
 \end{tabular}
 \end{flushleft}}

% ---------------------------------------------------------------------
% Macro for boxed ML functions, etc.
%
% Usage: (1) \begin{holboxed}\begin{verbatim}
%               .
%               < lines giving names and types of mk functions >
%               .
%            \end{verbatim}\end{holboxed}
%
%            typesets the given lines in a box.
%
%            Conventions: lines are left-aligned under the "g" of begin,
%            and used to highlight primary reference for the ml function(s)
%            that appear in the box.
% ---------------------------------------------------------------------

\newenvironment{holboxed}{\begin{flushleft}
  \begin{tabular}{@{}|c@{}|@{}}\hline
  \begin{minipage}[b]{\hsbw}
% \vspace*{-.55in}
  \vspace*{.06in}
  \begingroup\small\baselineskip\HOLSpacing}{\endgroup\end{minipage}\\ \hline
  \end{tabular}
  \end{flushleft}}

% ---------------------------------------------------------------------
% Macro for unboxed ML functions, etc.
%
% Usage: (1) \begin{hol}\begin{verbatim}
%               .
%               < lines giving names and types of mk functions >
%               .
%            \end{verbatim}\end{hol}
%
%            typesets the given lines exactly like {boxed}, except there's
%            no box.
%
%            Conventions: lines are left-aligned under the "g" of begin,
%            and used to display ML code in verbatim, left aligned.
% ---------------------------------------------------------------------

\newenvironment{hol}{\begin{flushleft}
 \begin{tabular}{c@{}@{}}
 \begin{minipage}[b]{\hsbw}
% \vspace*{-.55in}
 \vspace*{.06in}
 \begingroup\small\baselineskip\HOLSpacing}{\endgroup\end{minipage}\\
 \end{tabular}
 \end{flushleft}}

% ---------------------------------------------------------------------
% Emphatic brackets
% ---------------------------------------------------------------------

\newcommand\leb{\lbrack\!\lbrack}
\newcommand\reb{\rbrack\!\rbrack}


% ---------------------------------------------------------------------
% Quotations
% ---------------------------------------------------------------------


%These macros were included by ap; they are used in Chapters 9 and 10
%of the HOL DESCRIPTION

\newcommand{\inds}%standard infinite set
 {\mbox{\rm I}}

\newcommand{\ch}%standard choice function
 {\mbox{\rm ch}}

\newcommand{\den}[1]%denotational brackets
 {[\![#1]\!]}

\newcommand{\two}%standard 2-element set
 {\mbox{\rm 2}}

\begin{document}
\lstset{language=ML}
\maketitle{}

\begin{abstract}
This project is a part of HW7 of Assurance Foundations. The homework deals with integration of ML and HOL to \LaTeX.\ The goal of this report is to show reproducibility which is the groundwork for credibility that I have done this on my own without any external help. Every Chapter demonstrates the following sections:
	\begin{itemize}
		\item Problem Statement
		\item Relevant Code
		\item Test Results
	\end{itemize}
	
This project includes the following packages:
	\begin{description}
		\item[\emph{634format.sty}] A format style for this course
		\item[\emph{listings}] Package for displaying and inputting ML source code
		\item[\emph{holtex}] HOL style files and commands to display in the report
	\end{description}

This document also demonstrates my ability to :
	\begin{itemize}
		\item Easily generate a table of contents,
		\item Refer to chapter and section labels
	\end{itemize}

My skills and my professional details can be found at \url{https://www.linkedin.in/in/chiragsachdev}.
\end{abstract}

\section*{Acknowledgments}
I would gratefully acknowledge Dr. Shiu-Kai Chin and my other professors at Syracuse University and my Professors at Drexel University for being the wonderful mentors they are to guide me through my journey of obtaining a Master's Dregree.

\tableofcontents{}

\chapter{Executive Summary}
\label{cha:executive-summary}

\textbf{All requirements for this project are satisfied.}
Specifically,
\begin{description}
\item[Report Contents] \ \\
  Our report has the following content:
  \begin{enumerate}[{}]
  \item Chapter~\ref{cha:executive-summary}: Executive Summary
  \item Chapter~\ref{cha:11.6.1}: Exercise 11.6.1
    \begin{enumerate}[{}]
    \item Section~\ref{problem-statement-11-6-1}: Problem Statement
    \item Section~\ref{rel-code-11-6-1}: Relevant Code 
    \item Section~\ref{trans-11-6-1}: Session Transcripts
    \end{enumerate}
  \item Chapter~\ref{cha:11.6.2}: Exercise 11.6.2
    \begin{enumerate}[{}]
    \item Section~\ref{problem-statement-11-6-2}: Problem Statement
    \item Section~\ref{rel-code-11-6-2}: Relevant Code
    \item Section~\ref{trans-11-6-2} Session Transcripts
    \end{enumerate}
  \item Chapter~\ref{cha:11.6.3}: Exercise 11.6.3
    \begin{enumerate}[{}]
    \item Section~\ref{problem-statement-11-6-3}: Problem Statement
    \item Section~\ref{rel-code-11-6-3}: Relevant Code
    \item Section~\ref{trans-11-6-3}: Session Transcripts
    \end{enumerate}
 \item Appendix~\ref{cha:source-code-ex11-1}: Source Code Ex 11.6.1 and 11.6.2
 \item Appendix~\ref{cha:source-code-ex11-3}: Source Code Ex 11.6.3
  \end{enumerate}
\item[Reproducibility in ML and \LaTeX{}] \ \\
  The ML and \LaTeX{} source files compile with no errors.
\end{description}



%Ex 11.6.1
\chapter{Excercise 11.6.1}
\label{cha:11.6.1}

\section{Problem statement}
\label{problem-statement-11-6-1}
Add to the theory exTypeTheory by proving the following theorem. Save it within exType-
Theory.
[LENGTH_APP]
` $\forall$ l1 l2 . LENGTH (APP l1 l2 ) = LENGTH l1 + LENGTH l2

\section{Relevant Code}
\label{rel-code-11-6-1}
\begin{lstlisting}[frame=TBlr]
val APP_def =
Define
`(APP [] (l:'a list) = l) /\
(APP (h::(l1:'a list)) (l2:'a list) = h::(APP l1 l2))`;

val APP_ASSOC =
TAC_PROOF(
([],
``!(l1:'a list)(l2:'a list)(l3:'a list).
(APP(APP l1 l2) l3) = (APP l1 (APP l2 l3))``),
Induct_on`l1` THEN
PROVE_TAC[APP_def]
);
val _ = save_thm("APP_ASSOC",APP_ASSOC);

val LENGTH_APP = 
TAC_PROOF(
([], ``!(l1:'a list)(l2:'a list).
LENGTH (APP l1 l2) = LENGTH l1 + LENGTH l2``),
Induct_on`l1` THEN
ASM_REWRITE_TAC	[APP_def,LENGTH,ADD_CLAUSES]
);
val _ = save_thm("LENGTH_APP",LENGTH_APP);
\end{lstlisting}

\section{Session Transcript}
\label{trans-11-6-1}

\begin{session}
  \begin{scriptsize}
\begin{verbatim}
Definition has been stored under "APP_def"
val APP_def =
   |- (!(l :'a list). APP ([] :'a list) l = l) /\
   !(h :'a) (l1 :'a list) (l2 :'a list). APP (h::l1) l2 = h::APP l1 l2:
   thm
Meson search level: ................
Meson search level: ........
val APP_ASSOC =
   |- !(l1 :'a list) (l2 :'a list) (l3 :'a list).
     APP (APP l1 l2) l3 = APP l1 (APP l2 l3):
   thm
val LENGTH_APP =
   |- !(l1 :'a list) (l2 :'a list).
     LENGTH (APP l1 l2) = LENGTH l1 + LENGTH l2:
   thm
val it = (): unit
> 
*** Emacs/HOL command completed ***

> 
Process HOL finished
\end{verbatim}
  \end{scriptsize}
\end{session}


%Ex 11.6.2
\chapter{Excercise 11.6.2}
\label{cha:11.6.2}

\section{Problem statement}
\label{problem-statement-11-6-2}
Add to the theory exTypeTheory by defining the function Map (note that “M” is capitalized and “ap” is lowercase). The behavior of Map is illustrated by
$Map f [1;2;3;4] = [f1;f2;f3;f4]$
Prove the following theorem. Save it within exTypeTheory.
Map_APP
` Map f (APP l1 l2 ) = APP (Map f l1 ) (Map f l2 )
\section{Relevant Code}
\label{rel-code-11-6-2}
\begin{lstlisting}[frame=TBlr]
val Map_def =
Define
`(Map (f:'a -> 'b) [] = []) /\
(Map f (h::(l1:'a list)) = (f h)::Map f (l1:'a list))`;

val Map_APP =
TAC_PROOF(
([],`` Map f(APP l1 l2) = APP (Map f l1) (Map f l2)``),
Induct_on`l1` THEN
ASM_REWRITE_TAC[Map_def,APP_def]
);
val _ = save_thm("Map_APP",Map_APP);
\end{lstlisting}

\section{Session Transcript}
\label{trans-11-6-2}

\begin{session}
  \begin{scriptsize}
\begin{verbatim}
# # # # Definition has been stored under "Map_def"
val Map_def =
   |- (!(f :'a -> 'b). Map f ([] :'a list) = ([] :'b list)) /\
   !(f :'a -> 'b) (h :'a) (l1 :'a list). Map f (h::l1) = f h::Map f l1:
   thm
> > # # # # # <<HOL message: inventing new type variable names: 'a, 'b>>
val Map_APP =
   |- Map (f :'b -> 'a) (APP (l1 :'b list) (l2 :'b list)) =
   APP (Map f l1) (Map f l2):
   thm
> > > 
Process HOL finished

\end{verbatim}
  \end{scriptsize}
\end{session}



%Ex 11.6.3
\chapter{Excercise 11.6.3}
\label{cha:11.6.3}


\section{Problem statement}
\label{problem-statement-11-6-3}
Define a new theory nexpTheory corresponding to natural number expressions, which is similar to the example bexpTheory defined earlier on boolean expressions. Your implementation of nex-pTheory should have the following datatypes, definitions, and theorems.
\begin{enumerate}
\item The datatype nexp:
\\
nexp = Num num | Add nexp nexp | Sub nexp nexp | Mult nexp nexp
\\
\item A definition of the semantics of nexp expressions. Call this semantic function nexpVal. Remember, by
convention, the name of the defining theorem is nexpVal def. Make sure you follow this convention to avoid confusing HOL.
\item Prove and save the following theorems as part of your version of nexpTheory.
\end{enumerate}
\section{Relevant Code}
\label{rel-code-11-6-3}
\begin{lstlisting}[frame=TBlr]

val _ = new_theory "nexp";

val _ = Datatype
`nexp = Num num | Add nexp nexp | Sub nexp nexp | Mult nexp nexp`;

val nexpValDef =
Define
`(nexpVal (Num num) = num) /\
 (nexpVal (Add f1 f2) = (nexpVal f1) + (nexpVal f2)) /\
 (nexpVal (Sub f1 f2) = (nexpVal f1) - (nexpVal f2)) /\
 (nexpVal (Mult f1 f2) = (nexpVal f1) * (nexpVal f2))`;

(*Add 0*)
val Add0 = 
TAC_PROOF(
([],``!(f:nexp).nexpVal (Add (Num 0) f) = nexpVal f``),
Induct_on`f` THEN
ASM_REWRITE_TAC[ADD_CLAUSES,nexpValDef]
);
val _ = save_thm("Add0",Add0);

(*ADD SYM*)
val Add_SYM = 
TAC_PROOF(
([],``!(f1:nexp)(f2:nexp).nexpVal (Add f1 f2) = nexpVal (Add f2 f1)``),
Induct_on`f1` THEN
PROVE_TAC[nexpValDef, ADD_SYM]
);
val _ = save_thm("Add_SYM",Add_SYM);

(*SUB 0*)
val Sub_0 =
TAC_PROOF(
([],``!(f:nexp).(nexpVal (Sub (Num 0) f) = 0) /\
(nexpVal (Sub f (Num 0)) = nexpVal f)``),
Induct_on`f` THEN
PROVE_TAC[nexpValDef, SUB_0]
);
val _ = save_thm("Sub_0",Sub_0);


(*Multi ASSOC*)
val Mult_ASSOC = 
TAC_PROOF(
([],``!(f1:nexp)(f2:nexp)(f3:nexp).nexpVal (Mult f1 (Mult f2 f3)) =
nexpVal (Mult (Mult f1 f2) f3)``),
Induct_on`f1` THEN
ASM_REWRITE_TAC[MULT_ASSOC,nexpValDef]
);
val _ = save_thm("Mult_ASSOC",Mult_ASSOC);
\end{lstlisting}

\section{Test Case}
\label{trans-11-6-3}
\begin{session}
  \begin{scriptsize}
\begin{verbatim}

val _ = new_theory "nexp";

val _ = Datatype
`nexp = Num num | Add nexp nexp | Sub nexp nexp | Mult nexp nexp`;

val nexpValDef =
Define
`(nexpVal (Num num) = num) /\
 (nexpVal (Add f1 f2) = (nexpVal f1) + (nexpVal f2)) /\
 (nexpVal (Sub f1 f2) = (nexpVal f1) - (nexpVal f2)) /\
 (nexpVal (Mult f1 f2) = (nexpVal f1) * (nexpVal f2))`;

(*Add 0*)
val Add0 = 
TAC_PROOF(
([],``!(f:nexp).nexpVal (Add (Num 0) f) = nexpVal f``),
Induct_on`f` THEN
ASM_REWRITE_TAC[ADD_CLAUSES,nexpValDef]
);
val _ = save_thm("Add0",Add0);

(*ADD SYM*)
val Add_SYM = 
TAC_PROOF(
([],``!(f1:nexp)(f2:nexp).nexpVal (Add f1 f2) = nexpVal (Add f2 f1)``),
Induct_on`f1` THEN
PROVE_TAC[nexpValDef, ADD_SYM]
);
val _ = save_thm("Add_SYM",Add_SYM);

(*SUB 0*)
val Sub_0 =
TAC_PROOF(
([],``!(f:nexp).(nexpVal (Sub (Num 0) f) = 0) /\
(nexpVal (Sub f (Num 0)) = nexpVal f)``),
Induct_on`f` THEN
PROVE_TAC[nexpValDef, SUB_0]
);
val _ = save_thm("Sub_0",Sub_0);


(*Multi ASSOC*)
val Mult_ASSOC = 
TAC_PROOF(
([],``!(f1:nexp)(f2:nexp)(f3:nexp).nexpVal (Mult f1 (Mult f2 f3)) =
nexpVal (Mult (Mult f1 f2) f3)``),
Induct_on`f1` THEN
ASM_REWRITE_TAC[MULT_ASSOC,nexpValDef]
);
val _ = save_thm("Mult_ASSOC",Mult_ASSOC);
\end{verbatim}
  \end{scriptsize}
\end{session}

\appendix{}

\chapter{Source code: Ex 11.6.1 and Ex 11.6.2}
\label{cha:source-code-ex11-1}

\lstinputlisting{/home/csbd/Documents/COURSES/csbd/HW7/HOL/exTypeScript.sml}

\chapter{Source code: Ex 11.6.3}
\label{cha:source-code-ex11-3}
\lstinputlisting{/home/csbd/Documents/COURSES/csbd/HW7/HOL/nexpScript.sml}
\end{document}
